\label{Supported-Operating-Systems_003b}
\section*{Supported Operating Systems}
\index{Supported Operating Systems}
\index{Operating Systems}
\addcontentsline{toc}{section}{Supported Operating Systems}

\label{index-Supported-OSes-8}
\label{index-OSes-Supported-9}
Please note that due to the lack of Unix USB API standards, the USB code in
apcupsd works only on Linux and *BSD systems. In addition, at the current
release (3.10.17) the USB support for *BSD systems can at best be considered
BETA due to fragile kernel drivers. Drivers for other OSes can be written, but
it requires someone with a knowledge of the OS and the USB to do so.  (This
lack of a Unix USB API interface is one of the big failings of Unix.  It
occurs in other areas such as the GUI. Many people tout the diversity as an
advantage, but it is in fact a weakness.)  

The apcupsd maintainers develop it under Fedora (Red Hat); that port is,
accordingly, the most up to date and best tested.  There are enough Debian
Linux users that that port is also generally pretty fresh.  Slackware Linux is
also fully supported.  

apcupsd has also been ported to FreeBSD, NetBSD, OpenBSD, HP/UX, Solaris,
Alpha Unix and the Cygwin Unix emulation under Windows. It is quite likely to
work on those systems, though the port may occasionally get stale and require
minor tweaking.  apcupsd can also work on Unix-like systems, but without
USB mode.  apcupsd has been ported to OS~X/darwin with this limitation.

In 
\ilink{Operating System Specifics}{Operating-System-Specifics}
you'll find operating-system-specific tips for building and configuring
apcupsd.  
